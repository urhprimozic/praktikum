\documentclass[a4paper,12pt]{article}
\usepackage[slovene]{babel}
\usepackage[utf8]{inputenc}
\usepackage[T1]{fontenc}
\usepackage{lmodern}
% Paketi za matematiko

\usepackage{amsmath}  % razna okolja za poravnane enačbe ipd.
\usepackage{amsthm}   % definicije okolij za izreke, definicije, ...
\usepackage{amssymb}  % dodatni matematični simboli
\pagestyle{empty}
\begin{document}

\section*{Naloge iz matematike}

\begin{enumerate}    
\item Dokaži, da je enačba $(P\cap X)\cup (Q\cap X^c)=\emptyset$
rešljiva natanko tedaj, ko je $Q\subseteq P^c$.

\item Pokaži:
\begin{itemize}
\item $M=N\iff M+N=\emptyset$
\item $M=N=\emptyset \iff M\cup N = \emptyset$
\end{itemize}

\item Ali obstaja tak izjavni izraz $A$, da bosta izraza
$(p\wedge A)\vee (p\Rightarrow \neg A)$ in $(p \Rightarrow A)\Rightarrow q$ enakovredna?

\item Dokaži:
\begin{itemize}
    \item $(A \Rightarrow B) \sim (\neg B \Rightarrow \neg A)$
\end{itemize}

\item Poišči preneksno obliko formule $\exists x : P(x) \wedge \forall x: Q(x) \Rightarrow \forall x : R(x)$.

\item Vektorja $\vec{c} = \vec{a} + 2\vec{b} $ in $\vec{d} = \vec{a}-\vec{b}$
sta pravokotna in imata dolžino 1. Določi kot med vektorjema $\vec{a}$ in $\vec{b}$.

\item Določi definicijsko območje funkcije
%
\[
f(x) = \log \frac{x^2+1}{x^2-4x+3}   
\]
%
\item Izračunaj
%
\[
\cos ^2 \frac{3\pi}{8} + \cos ^2 \frac{5~pi}{8}+\cos ^2 \frac{7\pi}{8}+\cos ^2 \frac{8\pi}{8}  
\]
%
\item Dokaži, da za vsa naravna števila $n$ velja:
%
\[
\frac{1}{\sqrt{1}}+\frac{1}{\sqrt{2}}+\ldots+\frac{1}{\sqrt{n}} \ge \sqrt{n}   
\]
%
\item Naj bo $z$ kompleksno število, $z \neq 1$ in $|z|=1$.
Dokaži, da je število $i\frac{z+1}{z-1}$ realno.

\item Pokaži, da je funkcija $x \mapsto \sqrt{x}$ enakomerno zvezna na $[0,\infty)$.

\item Izračunaj limito
%
\[
\lim\limits_{x \to \infty} (\sin \sqrt{x+1}-\sin \sqrt{x})    
\]

\item Dani sta grupi $(G,*)$ in $(H,\circ)$. V množici $G \times H$ definiramo operacijo
%
\[
(g_1,h_1)\cdot (g_2,h_2)=(g_1*g_2,h_1 \circ h_2)   
\]
%
Pokaži, da je množica $G\times H$ grupa za to operacijo.

\item Pokaži, da ima $f(x) = 3x + \sin{(2x)}$ inverzno funkcijo in izračunaj $(f^{-1})(3\pi)$.

\item Izračunaj integral korenske funkcije
%Ti komentarji niso potrebni, a jih pišem, da ne pozabim na prazvo vrstico
\[
\int \frac{2 + \sqrt{x+1}}{(x+1)^2 - \sqrt{x+1}}dx    
\]
%( This comments don't have to be here, but I put them for a reminder. I dont want to remove the empty line.)
\item Krivulja je podana parametrično z enačbama
%
\[
    x(t)=\int_1^t \frac{\sin{u}}{u^2}du \qquad y(t)=\int_1^t \frac{cos{u}}{u^2}du
\]
%
Izračunaj dolžino poti od točke $x=0$ do točke, v kateri je tangenta prvič navpična.

\item Naj bo $\sum_{n=1}^\infty a_n$ absolutno konvergentna vrsta in $a_n \neq 1$ za $n \in \mathbb{N}$.
Dokaži, da sta vrsti
%
\[
    \sum_{n=1}^\infty \frac{a_n}{1+a_n} \qquad \text{in} \qquad \sum_{n=1}^\infty \frac{a_n^2}{1+a_n^2}    
\]
%
absolutno konvergentni.

\item Funkcijsko zaporedje $f_n: [a,b] \rightarrow [c,d]$ enakomerno konvergira na $[a,b]$ proti funkciji $f$.
Naj bo $g : [c,d] \rightarrow \mathbb{R}$ zvezna. Dokaži, da funkcijsko zaporedje $g \circ f_n$
enakomerno konvergina na $[a,b]$ in določi njegovo limitno funkcijo.

\item Izračunaj limito zaporedja
%
\[
\lim \limits_{n \to \infty} \frac{\sqrt[3]{n^2 + n -1}+\sqrt[3]{n}+n^2}{2n^2+\sqrt{n}+1}   
\]
%
\item Izračunaj $
\begin{pmatrix}
    1 & 2 & 3 & 4 & 5 & 6 \\
    4 & 5 & 2 & 6 & 3 & 1
\end{pmatrix}
^{-2000}
$

\item Poenostavi
%
\[
\frac{\dfrac{3+i}{2-2i}+\dfrac{7i}{1-i}}{1+\dfrac{i-1}{4}-\dfrac{5}{2-3i}}   
\]
%
\item Za dani zaporedji preveri, ali sta konvergentni
%
\[
a_n = \underbrace{\sqrt{2+\sqrt{2+\ldots+\sqrt{2}}}}_{n \text{ korenov}} \qquad b_n=\underbrace{\sin{(\sin{(\ldots \sin{1})})}}_{n \text{ sinusov}}
\]
%
\item Ugotovi, ali obstaja
%
\[
\lim \limits_{y \to 0} y \left(\frac{y+1}{y}-\sqrt{\frac{y^2+1}{y^2}} \right)
\]
%
Pomagaj si z limitama funkcije $\dfrac{x+1-\sqrt{x^2}+1}{x}$ v $-\infty$ in $\infty$.

\item Izračunaj naslednjo determinanto $2n\times 2n$, ki ima na neoznačenih mestih ničle.
%
\[
\begin{vmatrix}
    1 & & & 1 & & & \\
    & 2 & & 2 & & & \\
    & & \ddots & \vdots & & & \\
    1 & 2 & \cdot
\end{vmatrix}   
\]
%
\item Dana je funkcija
??
Določi parameter ?? tako, da bo ?? zvezna.
Izračunaj parcialna odvoda ?? in ?? za ??.
Izračunaj parcialna odvoda ?? in ??.
Če obstaja, izračunaj limito
??
Ali je funkcija ?? diferenciabilna?

\item Poišči vse rešitve enačbe
??

\item Dokaži binomsko formulo: za vsaki realni števili ?? in ?? in za vsako naravno število ?? velja
??

\item Naj bo
??
Pokaži, da je ?? podgrupa v grupi ??
neničelnih kompleksnih števil za običajno množenje.
Pokaži, da je ?? podgrupa v aditivni grupi ??
ravninskih vektorjev za običajno seštevanje po komponentah.
Pokaži, da je preslikava ??, podana s pravilom
??
izomorfizem grup ?? in ??.

\item Nariši grafe funkcij:
??

\end{enumerate}
\end{document}
