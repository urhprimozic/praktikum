% Predloga za prvi kolokvij iz praktikuma. Urh Primožič
% Velik delj predloga je povzet iz dokumentov na spletni učilnici FMF.
\documentclass[a4paper, 12pt]{article}
%options: <n>pt določa pisavo
%style: article, report, book

\usepackage[utf8]{inputenc} % vhodna datoteka je kodirana v sistemu UTF-8
\usepackage[T1]{fontenc}      % izhodna datoteka je kodirana v sistemu T1
\usepackage[slovene]{babel}   % uporabljamo slovenščino
%pisave:
\usepackage{lmodern}            % Boljši fonti, s pravimi šumniki
% \usepackage{times}            % Times New Roman
% \usepackage{palatino}         % Palatino
% \usepackage{concrete}         % Pisava, ki jo je uporabil Donald Knuth v knjigi "Concrete mathematics"

\usepackage{hyperref}            % Paket za povezave znotraj dokumenta in hiper-povezave

%\usepackage{url} % \url{url napišeš sem} naredi url v fancy pisavi. Priporočam uporabo hyperref namesto tega
\usepackage{amsmath}  % razna okolja za poravnane enačbe ipd.
\usepackage{amsthm}   % definicije okolij za izreke, definicije, ...
\usepackage{amssymb}  % dodatni matematični simboli
\usepackage{graphicx} % slike inb te zadeve
\usepackage{makeidx} % indexiran kazalo tujk
\usepackage{tikz} % fancy graik pa to
\usepackage{array} % komplicirane matrike in to

% Definicija okolij izrek, posledica
{\theoremstyle{theorem}
\newtheorem{izrek}{Izrek}[section]
\newtheorem{posledica}[izrek]{Posledica}
}

% Definicija okoli za definicije in vaje
{\theoremstyle{definition}
\newtheorem{definicija}[izrek]{Definicija}
\newtheorem{vaja}[izrek]{Vaja}
}



%Definicije komand
\newcommand{\mymacro}{Context of mymacro blabla}
\newcommand{\withargs}[2]{Uses #1 and uses #2}
\newcommand{\withdefvalues}[2][TheDefault]{The Default is #1, the other arg is #2}
%Vsi spodnji ukazi morajo biti znotraj math mode
\newcommand{\RR}{\mathbb{R}}
\newcommand{\CC}{\mathbb{C}}
\newcommand{\NN}{\mathbb{N}}
\newcommand{\contRR}[1][(\RR^2)]{\mathcal{C}#1}
%tole je mozn shit
\newcommand{\cont}[1][{}]{\mathcal{C} \ifthenelse{\equal{#1}{}}{}{#1}}
\newcommand{\contOn}[2]{([#1,#2])}
\newcommand{\vectArrow}[1]{\overrightarrow{#1}}
\newcommand{\vect}[1]{\mathbf{#1}}
\newcommand{\cmd}[1]{\texttt{\color{blue}{\textbackslash}#1}}
\newcommand{\demolenght}[1]{\rule{0.1pt}{5pt}\rule{#1}{0.1pt}\rule{0.1pt}{5pt}}

\newcommand{\tx}{\hspace*{0pt}\hfill\verb}



\title{Zapiski za {\LaTeX}} %Latex{} Nardi fany latex
% Zgleda da \LaTex{} nagaja. Uporabljaj {\LaTeX}
\author{Urh Primožič \and Andrej Bauer \and ostali} %Beseda fancy je tuklaj preveč v rabi
%Itak je vse prekopirano iz gradiva na SU in StackExchanga}

\begin{document}

    \maketitle
    %Aja, v tekstu se pazi praznih vrstic.
    \begin{abstract}
        Tak izgleda izvleček. Če je naslov napisan v \textbf{angleščini}, si verjetno pozabil paket babel.
    \end{abstract}

    \tx|\tableofcontents|
    \tableofcontents

    \newpage

    \section{Oblikovanje texta}
    %section* za brez številke
    \subsection{Osnovni ukazi za oblikovanje}
        \begin{itemize}
            \item \emph{poudarjeno besedilo} \tx|\emph|
            \item \textbf{krepko} \tx|\textbf{}|
            \item \emph{\textbf{In zdaj: \textnormal{običajno} znotraj česa } drugega} \tx|\textnormal{}| %Ampak ne delat tega
            \item \underline{Podčrtano} \tx|\underline{}| 
            \item \textsf{Sans serifna pisava} \tx|\textsf{}|
            \item \textsc{Male velike črke} \tx|\textsc{}|
            \item \textsl{Ležeča} \tx|\textsl{}|
            \item \texttt{Fiksna širina} \tx|\texttt|
        \end{itemize}
    \subsection{Velikosti pisav}
    Uporavljal \{ \}. 
    \begin{verbatim}
        {\Huge Tvoja mama homoseksualna}
    \end{verbatim}
    \begin{itemize}
        \item {\Huge Ogromno}  \tx|{\huge blabla}|
        \item {\Large Srednje preveliko} \tx|\Large|
        \item {\tiny Prefukam te za zajtrk majkemi} \tx|\tiny|
        \item Ostalo: \verb|Huge < hude < LARGE < Large < large < normalsize| 
        \item \verb|< footnotesize < scriptsize < tiny| 
    \end{itemize}

    \subsection{Narekovaji}
    \begin{itemize}
        \item `$({AltGr}+7)\cdot2$' \tx|`  text  '|
        \item ``Angleški narekovaji'' \tx|``  text  ''|
        \item  "> Slovenski tiskani"< \tx|">  text "<|
        \item "`Slovenski pisani"' \tx|"`  text  "'|
    \end{itemize}
    \subsection{Poravnave}
    \begin{center}
        Sredinska poravnava.\\
        \verb|\begin{center}|
    \end{center}
    \begin{flushright}
        Desna poravnava.\\
        \verb|\begin{flushright}|\\
        \end{flushright}
        \begin{flushleft}
                Simetrični ukaz za levo je \verb|\begin{flushleft}|
        \end{flushleft}
    \subsection{Povezave}
    \emph{Citati in literatura so spodaj}
        Namesto okolja \textbf{url} rajši uporabljam okolje \textbf{hyperref}, saj ta nudi interaktivne linke.
        Tako podaš povezavo do spletne strani:
        \begin{center}
            \url{https://github.com/urhprimozic} \tx|\url{  }|
        \end{center}
        Lahko pa podaš \href{https://github.com/urhprimozic}{povezavo, zamaskirano v besedilo.} \tx|\href{url}{besedilo}|
    \subsection{Naštevanje}
    Uporabljaj okolji \textbf{enumerate} in \textbf{itemize}
    \begin{enumerate}
        \item gnezdenje \tx|\item|
        \begin{enumerate}
            \item in zdj maš črke
            \item magija
        \end{enumerate}
        \item ponovno: magija
    \end{enumerate}
    Podobno obstaja okolje \textbf{descritpion}, ki je za moje pojme neuporabno.
    \begin{description}
        \item Pojdi domov 
        \item Jokaj
        \item Jokaj
    \end{description}


    \section{Matematika}
    Osnovna orodja nudi paket \textbf{amsmath}. Za dodatne simbole je potrebno poseči po \textbf{amssymb}. Definicije za okolja \emph{lema, dokaz (theorem), ...} nudi paket \textbf{amsthm}.
    \subsection{Prikazni in vrstični način}
    To, da naj bo $\epsilon$ poljubno realno število, večje od nič in to, da velja $\psi(x_i)=~\sum \limits_{n=0}^{f(x_i+1)}a_n$ lahko zapišeš v eni vrsti s pomočjo \$.
    \\ \tx|$\psi(x_i)=~\sum \limits_{n=0}^{f(x_i+1)}a_n$|
    Stvari lahko pišeš tudi v svoj odstavek. Torej:
    %
    \[
        \lim \limits_{x \Rightarrow \infty} \left( \underbrace{\sqrt[n]{x+\sqrt[n]{x+ \ldots + \sqrt[n]{x+\sqrt[n]t{n}}}}}_{n-\text{korenov}} \right) \qquad \cap \mathbb{A}=\zeta(\{1,2,3,4,5,6,7 \})
    \]
    %
    NE DELAJ PRAZNIH VRSTIC za matematiko. (Dobra praksa: zakomentiraj vrstico pred in za matematiko.)
    {\small
    \begin{verbatim}
        %
    \[
        \lim \limits_{x \Rightarrow \infty} \left( \underbrace{\sqrt[n]{x+\sqrt[n]
        {x+ \ldots + \sqrt[n]{x+\sqrt[n]t{n}}}}}_{n-\text{korenov}} \right) 
        \qquad \cap \mathbb{A}=\zeta(\{1,2,3,4,5,6,7 \})
    \]
    %
    \end{verbatim}}
    Seveda pa je vedno bolj zaželjeno uporabljati okolja AMS.
    \subsection{Okolja AMS}
    \subsubsection{Equation}
    Equation poravna formulo sredinsko. \verb|begin{equation*}| ne oštevilči enačbe.
    \begin{equation}
        \label{eq:vsota-kvadratov}
        x^2 + y^2 = 1
      \end{equation}
    \begin{verbatim}
        \begin{equation}
            \label{eq:vsota-kvadratov}  %S TEM SE BOŠ SKLICEVAL NA ENAČBO
            x^2 + y^2 = 1
          \end{equation}
    \end{verbatim}
    Klic na enačbo~\eqref{eq:vsota-kvadratov} dobiš z \tx|\eqref{ label }|
    \subsubsection{Gather}
    Z \texttt{gather} postavimo izraze enega pod drugega.
    \begin{gather}
        \log 2 = 1 - \frac{1}{2} + \frac{1}{3} - \frac{1}{4} + \cdots, \\
        \frac{2}{\frac{1}{x} + \frac{1}{y}} \leq \sqrt{x y} \leq \frac{x + y}{2}, \\
        \sum_{k = 1}^\infty \frac{1}{k^2} = \frac{\pi^2}{6}.
      \end{gather}
      Enako velja, da uporabi $*$ za brez oštevi+lčenja.
      \subsubsection{Multiline}
      Ko imaš rad bralca, mu daš izpeljavo čez več vrstic. To storiš z okoljem \texttt{multline}.Prva vrstica je
      poravnana levo, zadnja desno in vsem vmesne sredinsko.
      \begin{multline*}
        \sum_{i=1}^n x_i^2 \cdot \sum_{i=1}^n y_i^2
        - \left( \sum_{i=1}^n x_i y_i \right)^2 = \\
        \frac{1}{2} \cdot \sum_{i=1}^n x_i^2 \cdot \sum_{i=1}^n y_i^2
        +
        \frac{1}{2} \cdot \sum_{i=1}^n x_i^2 \cdot \sum_{i=1}^n y_i^2
        -
        \sum_{i=1}^n x_i y_i \cdot \sum_{j=1}^n x_j y_j = \\
        \frac{1}{2} \cdot \sum_{i,j=1}^2 x_i^2 y_j^2
        +
        \frac{1}{2} \cdot \sum_{i,j=1}^n x_j^2 y_i^2
        -
        \sum_{i,j=1}^n x_i y_j x_j y_i = \\
        \sum_{i,j=1}^n \frac{1}{2}\,
        \left(
          x_i^2 y_j^2 + x_j^2 x_i^2 - 2 x_i y_j x_j y_i
        \right)
        =
        \sum_{i,j=1}^n \frac{1}{2}\,
        \left(
          x_i y_j - x_j y_i
        \right)^2
        \geq 0.
      \end{multline*}
      \subsubsection{Align}
      Z okoljem \texttt{align} lahko poravnamo vrstice na določen znaku. Mesto, kjer morajo biti
      vrstice poravnane, označimo z znakom \texttt{\&}, prehod v novo vrsto označimo z \verb|\\|:
      %
\begin{align*}
    (x + y)^2 - (x - y)^2
    &= (x^2 + 2 x y + y^2) - (x^2 - 2 x y + y^2) \\
    \intertext{in zato}
    &= x^2 + 2 x y + y^2 - x^2 + 2 x y - y^2 \\
    &= 2 x y + 2 x y \\
    &= 4 x y.
  \end{align*}
  %
  Pa ne dat \verb|\\| na konc.
  \begin{verbatim}
    \begin{align*}
        (x + y)^2 - (x - y)^2
        &= (x^2 + 2 x y + y^2) - (x^2 - 2 x y + y^2) \\
        \intertext{in zato}
        &= x^2 + 2 x y + y^2 - x^2 + 2 x y - y^2 \\
        &= 2 x y + 2 x y \\
        &= 4 x y.
      \end{align*}
  \end{verbatim}
\end{document}